\documentclass[10pt,a4paper]{article}

\usepackage[utf8]{inputenc}
\usepackage[margin=1.2in]{geometry}
\usepackage[french]{babel}
\usepackage{amsmath}
\usepackage{amsfonts}
\usepackage{graphicx}
\usepackage{hyperref}
\usepackage[nottoc, notlof, notlot]{tocbibind}
\hypersetup{colorlinks=true,citecolor=black,filecolor=black,linkcolor=black,urlcolor=black}
\setlength{\parindent}{0.6cm} 
\setlength{\parskip}{0.10cm}
\usepackage[automark]{scrpage2}
\usepackage{color, colortbl}
\usepackage[table]{xcolor}
\usepackage{hyperref}


\pagestyle{scrheadings}

\ihead[]{Équipe Navigation}
\ohead[]{Plan de Développement Qualité}



\begin{document}
\pagestyle {plain}

\begin{titlepage}


\newcommand{\HRule}{\rule{\linewidth}{0.5mm}} 

\center

\textsc{\Large Université Paul Sabatier}\\[1cm] 
\includegraphics[scale=0.3]{figures/UPS.jpg}\\[0.6cm] 


\textsc{Master Intelligence Artificielle et \\ 
Reconnaissance des Formes \\ Master Robotique : Décision et Commande}\\[3cm] 

\HRule \\[0.4cm]
{ \huge \bfseries Rapport}\\[0.4cm] 
\LARGE Navigation Autonome de Robot Mobile

\HRule \\[1.5cm]
 

\begin{minipage}{0.4\textwidth}
\begin{flushleft} \large
\emph{Auteur:}\\
\href{mailto:thibaut.aghnatios@laposte.net}{Thibaut \textsc{Aghnatios} }  \\
\href{mailto:bouchetmarinee@gmail.com}{Marine \textsc{Bouchet} } \\
\href{mailto:bruno.dato.meneses@gmail.com}{Bruno \textsc{Dato} } \\
\href{mailto:klempka.tristan@gmail.com}{Tristan \textsc{Klempla} } \\
\href{mailto:lagoute.31@gmail.com}{Thibault \textsc{Lagoute} }  
\end{flushleft}
\end{minipage}
~
\begin{minipage}{0.4\textwidth}
\begin{flushright} \large
\emph{Tuteur:} \\
\href{mailto:lerasle@laas.fr}{Frédéric \textsc{Lerasle}}\\
\href{mailto:michael.lauer@laas.fr}{Michaël \textsc{Lauer}} \\
\href{mailto:taix@laas.f}{Michel \textsc{Taix}}
\end{flushright}
\end{minipage}\\[5cm]

\includegraphics[scale=0.3]{figures/laas.png} \\[1.1cm] 

\large 13 mars 2017
% laaaaaaaaaaaaaaaaaaaaaaaaaaaaaaaaaaaaaaaaaaaaaaaaaas
 

\end{titlepage}

\newpage


\subsection*{Suivi du document}

\begin{center}
    \begin{tabular}{| l | l | l | l | l |}
    \hline
     \rowcolor{gray} Nom du document & Version Majeure & Version Majeure & Date de création & Dernière version \\ \hline
    Rapport & A & 0 & 13/03/2017 & 13/03/2017 \\ \hline
    \end{tabular}
\end{center}


\subsection*{Auteurs du document}

\begin{center}
    \begin{tabular}{| l | l | l | l |}
    \hline
    \rowcolor{gray} Rédaction & Intégration & Relecture & Validation Interne \\ \hline
    Equipe & ?? & ?? & ?? \\ \hline

    \end{tabular}
\end{center}

\subsection*{Validation du document}

\begin{center}
    \begin{tabular}{| l | l | l | l |}
    \hline
     \rowcolor{gray} Validation & Nom & Date & Visa \\ \hline
    & & & \\
     \hline
    \end{tabular}
\end{center}

\subsection*{Liste de diffusion}

Le rapport du projet est diffusé à l'ensemble des clients et des intervenants externes aux projets.

\subsection*{Historiques de révision}

\begin{center}
    \begin{tabular}{| l | l | l | l |}
    \hline
     \rowcolor{gray} Version & Modification apportée & Auteur & Date \\ \hline
    A.0 & Création du document & Bruno Dato & 13/03/2017\\ \hline
     
    \end{tabular}
\end{center}

\newpage
\tableofcontents
\newpage
	

\section{Présentation du projet}
\label{sec:presentation}

\subsection{Contexte}

\subsection{Problématiques}

\newpage
\section{Recherche balle}
\label{sec:recherche_balle}

\newpage
\section{Navigation avec amers 2D dans un environnement connu}
\label{sec:navigation_avec_amers_2D_dans_un_environnement_connu}

\subsection{Solution mise en place}
\label{sec:solution_mise_en_place}

\subsection{Commande haut niveau}
\label{sec:commande_haut_niveau}

\subsection{Détection et localisation}
\label{sec:detection_et_localisation}


\noindent Avant toute chose, il est important de rappeler que quelque soit la situation :
\begin{itemize}
\item $/map \rightarrow /odom \rightarrow /base\_link $ (\url{cf. http://www.ros.org/reps/rep-0105.html})
\item /odom est le repère relatif, assimilé certain, du robot et est placé au départ sur /map
\item chaque frame peut avoir plusieurs fils mais qu'un seul parent
\item la transformé entre deux frames est décrite par deux attributs : 
  \begin{itemize}
  \item $m\_origin$ : Vector3 de translation 
  \item $m\_basis$ : Matrix3x3 de rotation
  \end{itemize}
\end{itemize}

\subsubsection{Détection}
\label{sec:detection}

Pour la localisation, nous utilisons des markers AR tags de la librairie $ar\_track\_alvar$. DEF 
Cette librairie a plusieurs avantages, x services 

\begin{itemize}
\item tracking
\item x ar
\item renvoie l'id du marker 
\item renvoie la position et l'orientation d'un amer dans le repère de la caméra $/camera\_rgb\_optical\_frame$
\end{itemize}

\begin{figure}
\center
\includegraphics[scale=0.6]{figures/artags.png} 
\caption{texte de la légende}	
\end{figure}

Dans notre projets

\subsubsection{Localisation}
\label{sec:localision}

\begin{itemize}

\item [localisation\_node.cpp] :

Ce nœud permet d'envoyer la nouvelle localisation du robot si il détecte un marqueur. 

\noindent La recherche se déclanche uniquement quand la commande haut niveau publie un $<$std\_msgs::Empty$>$ sur le topic $/nav/HLC/askForMarker$. \\

Dans le cas ou un marqueur est visible : \\
\noindent \underline{init} 
/map $\rightarrow$ /odom $\rightarrow$ /baselink $\rightarrow$ /camera\_rgb\_frame $\rightarrow$ /camera\_rgb\_optical\_frame \\
\noindent \underline{arrivée d'un marker} 
/camera\_rgb\_optical\_frame $\rightarrow$ /ar\_marker\_0 \\
\noindent \underline{traitement} \\
\indent /baselink $\rightarrow$ /camera\_rgb\_frame $\rightarrow$ /camera\_rgb\_optical\_frame $\rightarrow$ /ar\_marker\_0 \\
\indent on sait donc /ar\_marker\_0 $\rightarrow$ /baselink 
= /marker\_0 $\rightarrow$ /baselink \\
\indent sachant /map $\rightarrow$ /marker\_0, on sait /map $\rightarrow$ /baselink = /map  $\rightarrow$ /odom \\

\noindent Cette transformation est envoyé sur $/new\_odom$ que l'on publie avec le publisher $odom\_pub$. On publie également l'id du marqueur vu sur $/nav/loca/markerSeen$ sous un $<$std\_msgs::Int16$>$. 

Dans le cas où on ne voit rien on publie : -1 \\

\item [localisation\_broadcaster\_node.cpp] :

Ce nœud permet de tout le temps broadcaster la transfom entre /map et /odom. Il souscrit à la $<$geometry\_msgs::Transform$>$ /new\_odom, qui contient la "nouvelle" position certaine du robot. Dès que celle-ci est publiée, /odom actualise sa position et on réinitialise le nav\_msgs/Odometry : il devient notre nouveau référentiel.
\end{itemize}

\subsection{Commande}
\label{sec:commande}

\subsubsection{Odométrie}

\newpage
\section{Conclusion}
\label{sec:conclusion}

\newpage
\listoffigures
\newpage

\section*{ANNEXE}





\end{document}

