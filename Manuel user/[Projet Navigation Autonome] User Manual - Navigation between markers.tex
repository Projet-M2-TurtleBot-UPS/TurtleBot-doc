\documentclass[10pt,a4paper]{article}

\usepackage[utf8]{inputenc}
\usepackage[margin=1.2in]{geometry}
\usepackage[english]{babel}
\usepackage{amsmath}
\usepackage{amsfonts}
\usepackage{graphicx}
\usepackage{hyperref}
\usepackage[nottoc, notlof, notlot]{tocbibind}
\hypersetup{colorlinks=true,citecolor=black,filecolor=black,linkcolor=black,urlcolor=black}
\setlength{\parindent}{0.6cm} 
\setlength{\parskip}{0.10cm}
\usepackage[automark]{scrpage2}
\usepackage{color, colortbl}
\usepackage[table]{xcolor}
\usepackage{verbatim}


\pagestyle{scrheadings}

\ihead[]{Navigation Team}
\ohead[]{User Manual - Navigation Between Markers}



\begin{document}
\pagestyle {plain}

\begin{titlepage}


\newcommand{\HRule}{\rule{\linewidth}{0.5mm}} 

\center

\textsc{\Large Université Paul Sabatier}\\[1cm] 
\includegraphics[scale=0.3]{UPS.jpg}\\[0.6cm] 


\textsc{Master Intelligence Artificielle et \\ 
Reconnaissance des Formes \\ Master Robotique : Décision et Commande}\\[3cm] 

\HRule \\[0.4cm]
{ \huge \bfseries User Manual - Navigation Between Markers}\\[0.4cm] 
\LARGE Mobile Robot Navigation

\HRule \\[1.5cm]
 

\begin{minipage}{0.4\textwidth}
\begin{flushleft} \large
\emph{Authors:}\\
\href{mailto:thibaut.aghnatios@laposte.net}{Thibaut \textsc{Aghnatios} }  \\
\href{mailto:bouchetmarinee@gmail.com}{Marine \textsc{Bouchet} } \\
\href{mailto:bruno.dato.meneses@gmail.com}{Bruno \textsc{Dato} } \\
\href{mailto:klempka.tristan@gmail.com}{Tristan \textsc{Klempka} } \\
\href{mailto:lagoute.31@gmail.com}{Thibault \textsc{Lagoute} }  
\end{flushleft}
\end{minipage}
~
\begin{minipage}{0.4\textwidth}
\begin{flushright} \large
\emph{Tutors:} \\
\href{mailto:lerasle@laas.fr}{Frédéric \textsc{Lerasle}}\\
\href{mailto:michael.lauer@laas.fr}{Michaël \textsc{Lauer}} \\
\href{mailto:taix@laas.f}{Michel \textsc{Taix}}
\end{flushright}
\end{minipage}\\[5cm]


\large 30 March 2017
 

\end{titlepage}

\newpage


\subsection*{Document tracking}

\begin{center}
    \begin{tabular}{| l | l | l | l | l |}
    \hline
     \rowcolor{gray} Name & Major Version & Minor Version & Creation Date & Last version \\ \hline
    User Manual - Navigation & A & 1 & 30/03/2017 & 1/04/2017 \\ 
    Between Markers &  &  &  &  \\ \hline
    \end{tabular}
\end{center}


\subsection*{Document authors}

\begin{center}
    \begin{tabular}{| l | l | l | l |}
    \hline
    \rowcolor{gray} Redaction & Integration & Review & Validation \\ \hline
    Bruno Dato & Bruno Dato & ?? & ?? \\  \hline
    \end{tabular}
\end{center}

\subsection*{Document validation}

\begin{center}
    \begin{tabular}{| l | l | l | l |}
    \hline
     \rowcolor{gray} Validation & Name & Date & Visa \\ \hline
    & & & \\
     \hline
    \end{tabular}
\end{center}

\subsection*{Broadcast list}

User Manual - Navigation Between Markers is distributed to all clients and external stakeholders.

\subsection*{Review history}

\begin{center}
    \begin{tabular}{| l | l | l | l |}
    \hline
     \rowcolor{gray} Version & Additions or modifications & Author & Date \\ \hline
    A.0 & Document creation & Bruno Dato & 30/01/2017\\ \hline
    A.1 & Sections \ref{sec:Prerequisites} and \ref{sec:Navigation Between Markers} & Bruno Dato & 1/04/2017\\ \hline
    A.2 & Visibility & Thibaut AGHNATIOS & 02/04/2017\\ \hline
     
    \end{tabular}
\end{center}

\newpage
\tableofcontents
\newpage
	

\section{Prerequisites}
\label{sec:Prerequisites}

\subsection{Equipment}

\begin{itemize}
\item[•] TurtleBot 2
\item[•] AR markers ... (TODO)
\end{itemize}

\subsection{Software}

To be able to use any TurtleBot 2 with all the basic features, you need to complete the following tutorials :

\begin{itemize}
\item[•] \href{http://wiki.ros.org/turtlebot/Tutorials/indigo/Turtlebot%20Installation}{Turtlebot Installation} 
\item[•] \href{http://wiki.ros.org/turtlebot/Tutorials/indigo/PC%20Installation}{PC Installation} 
\item[•] \href{http://wiki.ros.org/turtlebot/Tutorials/indigo/Network%20Configuration}{Network Configuration} 
\end{itemize}


You also need the following software :

\begin{itemize}
\item[•] GIT \href{https://git-scm.com/download/linux}{[Installation]} 
\end{itemize}

\subsection{Workspace}

\subsubsection{Build workspace}

You need a ROS workspace (catkin workspace) to build our project before executing it. If you are running the ball search on the TurtleBot PC you have to create the workspace on the TurtleBot PC. In the case your are running it on a remote PC, you have to create the workspace on this PC.\

Place you where you want to build the workspace and execute the following commands :

\begin{itemize}
\item[]  \begin{verbatim}> mkdir -p /catkin_ws/src \end{verbatim}
\item[]  \begin{verbatim}> cd /catkin_ws/src \end{verbatim}
\item[]  \begin{verbatim}> catkin_init_workspace \end{verbatim}
\item[]  \begin{verbatim}> cd .. \end{verbatim}
\item[]  \begin{verbatim}> catkin_make \end{verbatim}
\end{itemize}

Then, in .bashrc, add the following lines (it's normal if some of them are already there) :

\begin{itemize}
\item[]  \begin{verbatim} #Initialisation Turtlebot kinect \end{verbatim}
\item[]  \begin{verbatim} export TURTLEBOT_3D_SENSOR=kinect \end{verbatim}
\item[]  \begin{verbatim} #ROS Version \end{verbatim}
\item[]  \begin{verbatim} source /opt/ros/indigo/setup.bash \end{verbatim}
\item[]  \begin{verbatim} source <YOUR_PATH>/catkin_ws/devel/setup.bash \end{verbatim}
\item[]  \begin{verbatim} #Select corresponding TurtleBot on your network \end{verbatim}
\item[]  \begin{verbatim} export ROS_MASTER_URI=http://<IP_OF_TURTLEBOT>:11311  \end{verbatim}
\end{itemize}

\subsubsection{Download package}

Now, you need to download the package containing the source code. Place you in your workspace (catkin\_ws), and execute the following commands :

\begin{itemize}
\item[]  \begin{verbatim}> cd src \end{verbatim}
\item[]  \begin{verbatim}> git clone https://github.com/Projet-Navigation-UPS/TurtleBot-pkgs \end{verbatim}
\end{itemize}

\subsubsection{Build executables}

Now that you have downloaded the source code, you just need to compile to build the executables files. Place you in your workspace (catkin\_ws) and run the command : 

\begin{itemize}
\item[]  \begin{verbatim}> catkin_make \end{verbatim}
\end{itemize}

Several red lines must appear in the compilation description, it means that the executables we need have been created.

\subsection{Map and markers configuration}

\subsubsection{Environment map}

You need yo create the map of the environment in which your navigating if it is not already available in the folder \textbf{/catkin$\_$ws/src/TurtleBot-pkgs/turtlebot$\_$proj$\_$nav/map}. To create the map, we use the \textbf{turtlebot$\_$navigation} package which provides a SLAM mode (\href{http://wiki.ros.org/turtlebot_navigation/Tutorials/indigo/Build%20a%20map%20with%20SLAM}{Tutorial link}). After turning on the TurtleBot and its laptop, execute the following commands the TurtleBot laptop :

\begin{itemize}
\item[]  \begin{verbatim}> roslaunch turtlebot_bringup minimal.launch \end{verbatim}
\item[]  \begin{verbatim}> roslaunch turtlebot_navigation gmapping_demo.launch \end{verbatim}
\end{itemize}

Then, on a remote computer, execute the visualization of the SLAM :

\begin{itemize}
\item[]  \begin{verbatim}> roslaunch turtlebot_rviz_launchers view_navigation.launch \end{verbatim}
\end{itemize}

To make the robot move and explore the environment, execute :

\begin{itemize}
\item[]  \begin{verbatim}> roslaunch turtlebot_teleop keyboard_teleop.launch --screen \end{verbatim}
\end{itemize}

Ounce the map is satisfying for the navigation, on another terminal you have to save it :

\begin{itemize}
\item[]  \begin{verbatim}> rosrun map_server map_saver -f <PATH>/catkin_ws/src/TurtleBot-pkgs/turtlebot\end{verbatim}
\item[]  \begin{verbatim}_proj_nav/map/my_map\end{verbatim}

\end{itemize}

\subsubsection{Markers disposition}
 
Within our project we have used  $16 \times 16$ markers placed 3 m away minimum from each other. We put their centers 31cm above the ground so the kinect-marker is as paralel as possible to the ground.

\subsubsection{Graph of the markers}

\subsubsection{Markers static transforms}

\subsubsection{Visibility map}

First, to generate the visibility map, it is necessary to previously have a map of the environment. This map is created virtually or by using the mapping available on the Turtlebot, used in the $ map\_server $.\\

The node $ visib\_pgmwriter\_node.cpp $ must not be modified, all configurations are done directly in the $ visib\_init.cpp $ file. Indeed, the node launches the function $ Writing\_map\_visib() $ that creates in a PGM file (Plain PGM: $ http://netpbm.sourceforge.net/doc/pgm.html\#plainpgm $) all markers defined in the $ graph.xml $ located in the $ / rsc $ folder. So for a given map size and for the configurations performed correctly in $ visib\_init.cpp $, just change the position and orientation of our markers in the $ graph.xml $ so that the new visibility map is automatically generated by running our $ visib\_pgmwriter\_node.cpp $ node again. \\

\begin{itemize}
\item[]  \begin{verbatim}> rosrun turtlebot_proj_launch visib_pgmwriter.cpp \end{verbatim}
\end{itemize}

For a different map scale, see the developer manual...

\newpage
\section{Navigation Between Markers}
\label{sec:Navigation Between Markers}

First, you need to turn on the TurtleBot (there is a switch button on the side of the robot base). Then, turn on the TurtleBot PC. We will now launch all the ROS nodes that we need to run our application.

\subsection{On the TurtleBot PC}

\subsubsection{Basic features}

If you are using the TurtleBot PC, open two terminals and chronologically execute the following commands to activate the minimal features and the vision features, one on each terminal :

\begin{itemize}
\item[]  \begin{verbatim}> roslaunch turtlebot_bringup minimal.launch \end{verbatim}
\item[]  \begin{verbatim}> roslaunch turtlebot_bringup 3dsensor.launch \end{verbatim}
\end{itemize}

\subsubsection{Navigation}

(à compléter)

\begin{itemize}
\item[]  \begin{verbatim}> roslaunch turtlebot_proj_nav navigation.launch \end{verbatim}
\end{itemize}

\subsection{On a remote PC}

\subsubsection{Basic features}

To execute the ball search from a remote PC, fist you have to ssh to the TurtleBot PC to launch the minimal and vision features. Open a fist terminal and write the following commands :

\begin{itemize}
\item[]  \begin{verbatim}> ssh turtlebot@<TURTLEBOTP_IP> \end{verbatim}
\item[]  \begin{verbatim}> roslaunch turtlebot_bringup minimal.launch \end{verbatim}
\end{itemize}

Then, in a second terminal :

\begin{itemize}
\item[]  \begin{verbatim}> ssh turtlebot@<TURTLEBOTP_IP> \end{verbatim}
\item[]  \begin{verbatim}> roslaunch turtlebot_bringup 3dsensor.launch \end{verbatim}
\end{itemize}

\subsubsection{Navigation}

(à compléter)

\begin{itemize}
\item[]  \begin{verbatim}> roslaunch turtlebot_proj_launch navigation.launch \end{verbatim}
\end{itemize}

\subsection{Behaviour of the navigation}

(à compléter)

\end{document}

